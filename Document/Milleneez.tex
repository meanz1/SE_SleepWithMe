\documentclass[conference]{IEEEtran}
\IEEEoverridecommandlockouts
% The preceding line is only needed to identify funding in the first footnote. If that is unneeded, please comment it out.
\usepackage{cite}
\usepackage{amsmath,amssymb,amsfonts}
\usepackage{algorithmic}
\usepackage{graphicx}
\usepackage{textcomp}
\usepackage{xcolor}
\usepackage{placeins}
\usepackage{caption}
\def\BibTeX{{\rm B\kern-.05em{\sc i\kern-.025em b}\kern-.08em
    T\kern-.1667em\lower.7ex\hbox{E}\kern-.125emX}}
\begin{document}

\title{Our sleepkeeper - Coco\\
{\footnotesize \textsuperscript{*}Note: Sub-titles are not captured in Xplore and
should not be used}
}

\author{\IEEEauthorblockN{Ko Eunseo}
\IEEEauthorblockA{\textit{Dept. of Information systems} \\
\textit{Hanyang University}\\
Seoul, Korea \\
koeunseo2@hanyang.ac.kr}
\and
\IEEEauthorblockN{Kim Minji}
\IEEEauthorblockA{\textit{Dept. of Information systems} \\
\textit{Hanyang University}\\
Seoul, Korea \\
99meanz@gmail.com}
\and
\IEEEauthorblockN{Moon MyeongKyun}
\IEEEauthorblockA{\textit{Dept. of Information systems} \\
\textit{Hanyang University}\\
Seoul, Korea \\
11lization@naver.com}
\and
\IEEEauthorblockN{Yun Haeun}
\IEEEauthorblockA{\textit{Dept. of Information systems} \\
\textit{Hanyang University}\\
Seoul, Korea \\
haeun2138@hanyang.ac.kr}

}

\maketitle
\begin{abstract}
Modern people habitually watch videos through the media before sleeping. The Corona Blue phenomenon caused by COVID-19 intensified people's irregular sleep patterns, and more people couldn't sleep before going to bed and watched videos until late at night. Our team aimed to create an application that automatically provides people with the necessary functions before and after sleep, targeting LG's mobile monitor "Stand by me." Our application not only provides video recommendations and power control so that users can fall asleep, but also provides a function to maintain regular sleep time. Therefore, this helps users maintain their lifestyle. And we maximized the convenience of users by automating the functions of the application.
\end{abstract}
\FloatBarrier
\begin{table}[]
\vspace{-30\baselineskip}
\caption*{Role Assignment}
{\normalsize
\resizebox{\columnwidth}{!}{%
\begin{tabular}{|p{2cm}|p{2cm}|p{6cm}|}
\hline
Rolese              & Name            & Task description and etc.      \\ \hline
User/Customer       &  Moon Myeongkyun & Find out what kind of sleep patterns people who use electronic devices such as TVs and smartphones usually have before going to bed. Investigate what functions people with these lifestyle patterns can use mainly and whether they can help people with insomnia. \\ \hline
Development Manager & Yun Haeun               & The development manager oversees the overall progress of the project. Prepare minutes and manage the schedule about the progress of the project. It also sets future goals for our team and plays a role in monitoring whether each job is performing well.                                                                \\ \hline
Software Engineer   & Kim Minji               &  Software developers think about software systems. Investigate the software necessary to implement and think specifically about how to implement it. The best way to connect the application with the speaker-equipped “Stand by me” is to be found. With feedback from the development manager, continuous updates should be made so that it can be software that users can satisfy.   \\ \hline
Designer            & Ko Eunseo                & The designer produces our team's logo and icon. 'Figma' will be used to design the overall app UI. In the final stage of manufacturing, 3D modeling and data analysis results reports for electronic devices will also be designed together.  \\ \hline
\end{tabular}%
}
\end{table}
\FloatBarrier
\vspace{3\baselineskip}
\section{Introduction}
\vspace{1\baselineskip}
\subsection{Motivation}
Bedrooms are said to be the most important space for modern people, including the Mz generation, at home. However, as COVID-19 spreads, irregular life patterns become commonplace due to social isolation, and so-called "Corona Blue" is becoming a problem. Because of this, many people are staying up all night.

Modern people use electronic devices such as smartphones in their bedrooms before going to bed.It is known that the habit of using electronic devices before going to bed interferes with sleep. However, there are many people who can't stop even though they know this. This is because there are many people who can't stop watching their smartphones before going to bed, thinking that it's better to look at their smartphones than tossing and turning while they can't sleep.

People use electronic devices before going to bed, but it usually interferes with sleep. But what about electronic devices that help me sleep, rather than interfere with sleep? In fact, many people recognize the need for a good night's sleep, so they look up videos that help them sleep on YouTube or install and use sleep management apps.

However, smartphones are not suitable for the purpose of sleep management because they have to be held and viewed directly on a small screen. It is also inconvenient to use the smartphone until it gets overheated and then charge it all night again for the next morning.

By improving these points, our team devised Sleep with Me, which utilizes the advantages of LG's mobile TV Stand by Me. "Sleep with me" is an app linked to "Stand by me," which helps people get a comfortable sleep and have regular sleep patterns. In this paper, we will look at the main customer base of "Sleep with me" and the current market status that supports Sleep-tech. Based on this, I will explain the main role of "Sleep with me".
\vspace{1\baselineskip}
\subsection{Intended Audience \& Use}
Currently, the customer base of "Stand by me" is mainly composed of MZ generation and newlyweds. In particular, "Stand by me" is a mobile TV, so you can enjoy video content anywhere in the house. In addition, it is cheaper than general TVs, so both primary and secondary pre-orders have been sold out, especially among MZ generation consumers.

In this way, we looked at the needs of the main customer base of "Stand by me," which succeeded in business feasibility. As mentioned in 1.1, we focused on 1) "Stand by me"'s property that can be used until sleep because it is mobile TV, 2) the majority of modern people's main TV viewing time is late at night after work, 3) not only those who are currently suffering from insomnia, but also those who want to be guaranteed regular sleep time even if they sleep well. Therefore, we set the user keyword of "Sleep with me" as "sleep". Since Sleep with me is an app that is linked to Stand by me, which is mainly used at home, end-user will be individuals at home.
\vspace{1\baselineskip}
\subsection{Existing Market}
As ‘sleep-tech’ spreads, various devices and applications for high-quality sleep are being released. By analyzing the advantages and disadvantages of sleep-tech currently provided in the market, we selected the functions to be applied and supplemented in our 'Sleep with me'. Current market trends are as follows.
\vspace{1\baselineskip}
\subsubsection{Non-wearable device}
\paragraph{Representative product}
'Sleep Sense' is a thin and long plate-type device manufactured by Samsung Electronics and Israel's IoT healthcare venture. If user put the device next to the bed, user can analyze the pulse, breathing, and movement that occur during sleep in real time and provide information on smartphone.
\paragraph{Advantages}
The device and smartphone are interlocked, making it convenient for users to  use.
\paragraph{Disadvantages}
If non-wearable devices simply provide functions for 'sleep', the number of users will inevitably be small compared to the price.
\paragraph{Application to ideas}
‘Coco’ should choose a way to induce sleep without compromising the unique functions of LG Electronics' home appliances.
\vspace{1\baselineskip}
\subsubsection{Wearable device}
\paragraph{Representative product}
'Smart Sleep Headband' is a sleep inducing product provided by Philips.When a user wears the device on his or her head, it helps to relieve sleep and tension through white noise. When linking with the headband application, the user's sleeping time and waking time are identified to manage the living pattern.
\\
‘AMO+’ is a wearable device for sleep that is worn in the form of a necklace. By applying ultra-fine strength and frequency electromagnetic signals to the human body in a non-contact manner, imbalance in the body is improved to improve sleep quality.
\paragraph{Advantages}
The wearable device is in the form of being worn or attached to a human body. Therefore, sleep can be measured with high accuracy, but poor wearability can interfere with sleep.
\paragraph{Disadvantages}
The wearable device is in the form of being worn or attached to a human body. Therefore, poor wearability can interfere with sleep.
\paragraph{Application to ideas}
'Coco' we will propose should pursue accuracy and think of a way that does not interfere with human sleep as much as possible.
\vspace{1\baselineskip}
\subsubsection{Application}
\paragraph{Representative product}
: ‘Calm' is an app that helps meditation and sleep. User can induce sleep by listening to the stories he want (the stories of celebrities, fairy tales, etc.) along with the sounds of nature (rainy sounds, forest sounds, waterfall sounds, etc.). User can set as much time as you want with a timer. It also provides meditation guides for users' mental and health, and music for relaxation and sleep.
\paragraph{Advantages}
The app is convenient to use because it has a lower price barrier than the device. It is convenient to use because it is a non-contact method.
\paragraph{Disadvantages}
Since the app is used in mobile phones, the user must hold the mobile phone himself to watch the video in the application, and the user can feel frustrated with a small screen. In addition, you may feel uncomfortable such as heat generation or battery consumption due to the fact that you charge it all night and use it all day the next day.
\paragraph{Application to ideas}
Our idea is to set the main device as a TV, not a mobile phone. In addition, based on the fact that 'Calm' provides various contents, we intend to provide various sleep-inducing contents in the apps we will provide.
\vspace{1\baselineskip}
\subsection{Goal}
\subsubsection{It presents ideas using TV, the most commonly used LG home appliance just before going to bed}
Most people fall asleep watching TV. Focusing on this point, we would like to propose ‘Sleep with Me’, which can help induce sleep and identify sleep patterns on TV without using wearable devices to prevent disturbance to sleep.
\vspace{1\baselineskip}
\subsubsection{It is intended to identify the user's sleep pattern by identifying the starting and ending points of the sleep time before bedtime}
It provides a UI that allows users to check their sleep patterns through ‘Sleep with me’, collect data, and easily check the data so that they can regularly manage and grasp their sleep patterns.
\vspace{1\baselineskip}
\subsubsection{The function is automatically terminated during sleep time so that 'Sleep with me' does not interfere with sleep}
If the TV monitor is operated while users are sleeping, it may interfere with sleep. Therefore, after being counted as the sleep start time, I would like to propose an automatic turn-off mode that can be automatically turned off without the user turning off the TV or setting the time himself.
\vspace{1\baselineskip}
\subsubsection{Taking advantage of TV's advantages, AI technology and IoT are utilized using monitor screens and sounds}
‘Sleep with me’ supports personalized video services that can induce sleep and enables language commands through natural language recognition. In addition, when linking with a mobile phone application through an IoT function, it provides convenience to enable alarm service and mode setting on a mobile phone.

\vspace{2\baselineskip}
\section{Requirement Analysis}
\vspace{1\baselineskip}
\subsection{Profile}
Due to the characteristics of home appliances, one device may have multiple users, so it is possible to create a personal profile that can use their own settings. The information you need to enter when creating a profile is the profile name, age, gender, desired sleep time, and preferred category. The created profile can be checked on the profile page where all profiles are gathered, and can be freely deleted. In the initial state where no user has created a profile, they are logged in with the default profile. In the case of a multi-person household, you can create a new profile and change the profile.
\vspace{1\baselineskip}
\subsection{Sleep mode}
When the user presses the “Sleep Mode” button (or speaks through the speaker) as a trigger, the sleep mode is activated. When sleep mode is active, a translucent moon-shaped mark appears in the upper left corner so that you can check it. Most of the functions to be described later operate on the premise that sleep mode is activated.
\vspace{1\baselineskip}
\subsection{Recommend and play videos that are good for sleeping}
"Sleep with me" recommends a list of videos to help you fall asleep by encouraging the user to press the "sleep video" button or speak through the speaker. A user can select a category of videos, and when a desired video is selected among them, a playlist related to the video is automatically created. If you do not like the video while watching, you can move to the next video by pressing the 'next video' button (or speaking through the speaker) as a trigger, and the list of videos the user watched until the end is recorded. By using this history to train your application, you can generate a highly accurate list of recommended videos. If the user wants to continue watching the desired video, this function does not work. For example, if the user wants to watch a movie or media transmitted from a platform other than Sleep Custom Videos, the sleep mode will remain and the video recommendation will not work.
\vspace{1\baselineskip}
\subsection{Exit stand by me screen and record bedtime}
From the time the sleep mode is activated, a pop-up window appears at the top every 20 minutes. The pop-up window contains the text ‘Are you still awake?’ and an ‘OK’ button. The user can dismiss the pop-up window in two ways before the next pop-up window appears. The first way is to click the ‘OK’ button. The second method is a method in which the pop-up window disappears when the user's voice is recognized through the speaker. If the next pop-up window appears while the pop-up window already exists, it is assumed that the user is sleeping. The time when the user's sleep is confirmed is recorded as the bedtime, the video played in "Stand by me" ends, and the screen is turned off.
\vspace{1\baselineskip}
\subsection{Customized sleep time alarm}
The user sets his or her sleeping time through the app in advance. It is based on manually setting an individual's optimal sleep time, but in the setting process, “Stand by me” recommends optimal sleep time according to gender and age to encourage users to be guaranteed optimal sleep time. When the sleep time is set, the time to wake up from the time when the user's bedtime time is recorded is automatically calculated to make the alarm sound in “Stand by me”. When the alarm goes off, two buttons appear on standby: a button to end the alarm and a button containing the meaning that the user has woken up. When the alarm end button is pressed, the sleep time alarm function of Standby Me is terminated as it is. When the wake-up button is pressed, the alarm function is terminated, and the wake-up time of the user is recorded to store the sleep time of the user.
\vspace{1\baselineskip}
\subsection{Sleeping pattern GUI provided}
“Stand by me” provides a GUI that allows you to see at a glance whether the desired sleep time has been reached on a daily, weekly, monthly, or yearly basis through the user's stored sleep record data. It helps users intuitively know their sleep patterns by marking them green on the day they reach the optimal sleep time they set, red on the day they do not reach, and gray on the day they do not use “Stand by me” ‘s sleep mode. If you press the moon-shaped translucent mark when activating the sleep mode, you can see the user's sleep pattern GUI at any time.
\\
\\
\\
\section{Development Environment}
\vspace{1\baselineskip}
\subsection{Choice of software development platform}
Our team will use webOS, a TV software development operating system used by LG Electronics, and Android studio for application development. WebOS will be used to implement an automatic TV off service that will work on ‘Stand by Me’ and will work on linux-based virtual machines. Android studio is used to implement our apps, including video recommendation services using YouTube api. We will demonstrate the behavior of our services and apps at ‘Stand by Me’ using TV and tablet emulators supported by each operating system. The tools used for planning are figma and AdobeXD (2021), and the languages used for development are JAVA, html, css, javascript (JS).
\vspace{-30\baselineskip}
\FloatBarrier
\begin{table}[]
\vspace{-30\baselineskip}
{\normalsize
\resizebox{\columnwidth}{!}{%
\begin{tabular}{|p{2cm}|p{7cm}|}
\hline
Tool and language            & Reason   \\ \hline
JAVA       &  The biggest feature of JAVA is that it is an independent language of a platform. The program made of java works without any problems if only jvm is installed for the platform. Also, the stability is excellent in two aspects. First, since it is a popular language, there are many references and open sources, and based on this, many large projects have been carried out, so stability is guaranteed in many areas. And secondly, it does not allow pointer variables or memory direct access functions, and it does not allow multiple inheritance, so it is highly stable. JAVA is used as a major language when developing in Android studio. \\ \hline
HTML/CSS /Javascipt(JS) & HTML(hypertext markup language) is a markup language for making web pages. Constructive structuring to create web documents is an important role, and because it has images and multiple objects embedded therein, it is easy to create web documents using "tags". CSS (cascading style sheets) is a style sheet language that uses html elements to define how they look in various media and adds design elements to structured html documents. Javascript is an object-based script programming language that is mainly used within a web browser. It allows web pages to operate by dynamically changing multiple elements and content. We create an external web app based on webOS using html/css/javascript. 
\\ \hline
\end{tabular}%
}
\end{table}
\FloatBarrier
\subsection{Software in uses}
\subsubsection{Android studio(2020.3.1)}
Android Studio is based on IntelliJ IDEA and is an official integrated development environment (IDE) for developing android apps. The android studio features IntelliJ's code editor and developer tools, and supports flexible gradient-based build systems, fast and functional emulators, integrated environments that can be developed for all android devices, code templates and github integrations, extensive test tools and frameworks, tint tools, C++ and NDK support, and Google cloud platform. These features increase productivity when building the android app on ‘Stand by Me’
\vspace{1\baselineskip}
\subsubsection{Github}
GitHub is a web service that supports git store hosting. A git is a distributed version management system and instruction to track changes in computer files and coordinate the operations of those files among multiple users. GitHub supports these feathers in the web format so that they can be easily viewed as a graphical interface. GitHub allows colleagues working on a project to share one workspace and set up an environment optimized for collaboration using the functions of git such as merge, commit, branch, etc. We will also share one repository to collaborate efficiently.
\vspace{1\baselineskip}
\subsubsection{webOS IDE/CLI/Emulator}
WebOS is a mobile operating system running on the Linux kernel that started at Palm and is currently being developed by LG Electronics. Since we use webOS 6.0 on our target device ‘Stand by me’, we implement some of the functions using webOS. 
Here, webOS IDE (Integrated Development Environment) is a software for building an application that combines common developer tools into one GUI (Graphical User Interface (GUI). WebOS also provides IDE to support the graphical development environment. We generally write codes in IDE.
The webOS CLI (Command-line Interface) is a method in which users and computers interact with commands through terminals or virtual terminals. Using terminals, we use the app and service package and app distribution to create ipk files as cli. Also, we use the are-inspect command to float a console to find out how the app we make works.
WebOS Emulator is a tool that indirectly shows how the software we make works on the target device. Several devices equipped with webOS can quickly enter the app menu and multitask without disturbing the screen you are watching through the card view, which is an advantage of webOS. ‘Stand by Me’ (or a device equipped with webOS) also allows you to preview how our app Sleep with me works with the card view function.
\vspace{1\baselineskip}
\subsubsection{NUGU developer}
NUGU is Sk Telecom's artificial intelligence platform that understands users' natural language requests to understand their intentions and provides information or related services. NUGU developers can provide AI services to NUGU, devices, and app users based on technologies such as voice recognition, speech synthesis, and understanding of natural languages. NUGU developers support NUGU play kits that help NUGU's PoC easily provide new features and services by Play, and NUGU biz kits that allow Play to be used privately for business purposes and send Announcement messages. It also provides NUGU SDK that help you directly provide NUGU's various services on various devices or applications.
\vspace{1\baselineskip}
\subsubsection{Virtual box}
As a computer virtualization program, user can use most of the existing OSs with this program. It offers hardware virtualization VT-x from Intel and AMD-V from AMD. Hardware parts can be implemented virtually. Hard disks are emulated in a container format called VDI. Currently, many emulators using virtual box are used, and in our project, one of them, webOS TV was used.
\vspace{1\baselineskip}
\subsubsection{Figma}
Figma is a UI GUI production tool and a platform for design. Figma is suitable for efficient prototype work because it is also available on web browsers and free of charge. In addition, there is a "development tool bar" that conveniently helps develop html and css, so we would like to use the tool as the main design tool. In collaboration, a prototype will be made with pigma, and the account holder will share the link to check the art board and work online at the same time.
\vspace{1\baselineskip}
\subsubsection{Adobe XD 2021}
Adobe XD is one of the design tools supported by Adobe that is optimized for UI GUI production. Before using Figma, we will use XD for prototype work. The XD has a sample screen that supports the TV screen, so it will be conveniently produced using the tool.

\section{Specifications}
Our app is divided into functions implemented on the TV itself and functions implemented on the app. Page 1.1, page 2.2, page 4.1, and page 4.2 are functions that run outside the app, and the rest are screens that run on the app.
\FloatBarrier
\begin{table}[]
\vspace{-30\baselineskip}
{\normalsize
\resizebox{\columnwidth}{!}{%
\begin{tabular}{|p{1.5cm}|p{7.5cm}|}
\hline
Page Num. & Explanation   \\ \hline
1 &  How to run our app and functions at first \\ \hline
2 & The two pages that come out when you want to run our app \\ \hline 
3 & Account creation page\\ \hline
4 & ‘Sleep mode’ execution page (TV turned off automatically)\\ \hline
5 & Video recommendation page\\ \hline
6 & Alarm settings page\\ \hline
7 & Sleep data check page
\\ \hline
\end{tabular}%
}
\end{table}
\FloatBarrier
\subsection{page 1.1}
\centerline{\includegraphics[width=250px]{00.png}}
The ‘Stand by Me’ currently has built-in buttons such as sound mode and power saving mode as side bars. Add a button to this button screen that allows you to turn on and off 'sleep mode'. Users can manually turn on and off 'sleep mode' at any time they want, and when 'sleep mode' is turned on, go to page 2.1. When the alarm function of our application ‘Sleep with Me’ is terminated with ‘Sleep Mode’ turned on, ‘Sleep Mode’ automatically turns off. You can access 'Sleep with Me' through the application on the “Stand by Me’ home screen.
\subsection{page2.1}
\centerline{\includegraphics[width=250px]{01.png}}
Page 2.1 is the home screen of the application. A greeting phrase containing the user's name is displayed in the center of the top of the page. And in the upper right corner, there is a ‘sleep mode’ on/off button in the shape of a moon. The button brightens when the 'sleep mode' is on, and the button darkens when the 'sleep mode' is off. The four buttons in the center are marked with 'account', 'recommendation', 'alarm', and 'My DATA', and are connected to pages 3.1, 5.1, 6.1, and 7.1, respectively.
\subsection{page2.2}
\centerline{\includegraphics[width=250px]{02.png}}
When the user turns on 'sleep mode', The alarm goes up, which is “This is 00's ‘sleep mode’. Do you want to start ‘sleep mode’?” under the name of the last user used. If the name of the user who currently uses ‘sleep mode’ and the user applied to the notification are the same, press the "Yes" button on the left side of the bottom of the notification to execute "sleep mode." If the name of the user currently using 'sleep mode' and the user applied to the notification are different, you can press the 'No, I'm a different person' button on the bottom right of the notification. By pressing this button, you can go to the account page 3 of the sleep with me application and add or change your account. After changing, turn on the 'sleep mode' again through the button on the left side of page 1.1 and use subsequent functions.
\subsection{page3.1}
\centerline{\includegraphics[width=250px]{03.png}}
If you press the account icon on the home screen on page 2.1, or press the "No, I'm someone else" button on page 2.2, a page appears to set up your account. On the Account Settings page, users can change their accounts by selecting their accounts among multiple accounts. Each of the accounts appears on the screen with a circle icon, and the account currently in use is displayed with a purple border. In addition to the account icon, pressing the '+' icon leads to the page where you add a new account. You can return to page 2.1 by pressing the home button at the top left.
\subsection{page3.2}
\centerline{\includegraphics[width=250px]{04.png}}
If you press the '+' icon on page 3.1, you will see a screen to add a new account. On this page, users can create an account by entering their name, gender, age, and taste information. Taste information is reflected in the video recommendation after you select a topic that you prefer among the six topics presented. You can choose multiple topics and up to three. Enter all the information and finally press the "Complete" button at the bottom right to create an account. If you want to cancel the account creation, the user returns to page 3.1 by pressing the "Cancel" button on the bottom left.
\subsection{page4.1}
\centerline{\includegraphics[width=250px]{05.png}}
When the user starts ‘sleep mode’ on page 2.2, a notification will appear on the screen to check whether the user is sleeping or not every certain period specified by the user. At this time, since the notification operates on the system itself, the user can receive notifications even if they are doing other actions. You can set the notification cycle on notification page 6.1 and the default value is 20 minutes. If you press the ‘Yes’ button to the ‘Are you going to see more?’ notification disappears, and the notification will appear repeatedly according to the notification cycle from the time you press the button.
\subsection{page4.2}
\centerline{\includegraphics[width=250px]{06.png}}
If there is no response more than 20 minutes (set time) to the notification, ‘Are you going to watch more?’ the screen darkens and saves the start time of bedtime.
\subsection{page5.1}
\centerline{\includegraphics[width=250px]{07.png}}
When the user clicks the ‘Video recommendation’ button on page 2.1, or the speaker asks “Can I play a video to help you fall asleep?” If you answer “yes” to this question, the video recommendation page 5.1 appears. On the recommendation page, there are buttons for a total of 3 items: ASMR, Meditation, and Music. Users can navigate to page 5.2 of that topic by clicking on one of these buttons. This action can also be performed through the speaker, and the user just speaks the words ‘ASMR’ or ‘meditation’ or ‘music’ into the speaker. You can return to page 2.1 by pressing the home button in the upper left corner. This action can be performed by the user through the artificial intelligence speaker, and the user "go home." Alternatively, you can say the sentence “Turn off the app” to the speaker.
\subsection{page5.2}
\centerline{\includegraphics[width=250px]{08.png}}
In page 5.1, the selection made by the user is indicated in the upper left corner. In the center, the videos reflecting the user's preference are listed. The application uses the user's information (age, gender, taste) and viewing history to find and play suitable videos on YouTube. You can return to page 2.1 by pressing the ‘Home’ button in the upper left corner, and the ‘Back’ button in the lower left to return to page 5.1. This action can also be performed through the speaker, and the user "go home." Just say the sentence to the speaker.
\subsection{page6.1}
\centerline{\includegraphics[width=250px]{09.png}}
Page 6.1 allows you to set an alarm at any time you want. A total of four settings can be made on the page: "Target Sleep Time," "Minimum Weather Time," "Repeatability," and "Sleep Check Cycle." Alarms are set based on time, not on time. 1) When the "target sleep time" is set, the alarm goes off after the "target sleep time" from the sleep start time recognized by Standby Me. 2) The "minimum wake-up time" means the time you have to wake up regardless of the "target sleep time." If you set the "minimum wake-up time," the alarm will go off at that time even if the "target sleep time" is not reached. 3) You can set whether to repeat the alarm or not by setting whether to repeat it. 4) The "sleep status check period" is a setting value that allows you to change the notification period for checking sleep status on page 4.1. The cycle can be changed every 5 minutes, and the default value is 20 minutes.

\subsection{page6.2}
\centerline{\includegraphics[width=250px]{10.png}}
Page 6.2 is the page when the alarm goes off. The alarm screen displays the time, date, and day of the week. There are two buttons on the screen: "awake" and "sleep" Press the "awake" button to end the alarm. If you press the "sleep" button, if you set the alarm repetition, the alarm will go off again after the set repetition cycle. If you haven't set the alarm repetition setting, the alarm will end immediately.
\subsection{page7.1}
\centerline{\includegraphics[width=250px]{11.png}}
Page 7.1 is the page that shows the user's sleep time data at a glance. The left icon table shows whether the 'target sleep time' set in Alarm Page 6.1 has been achieved. ‘Sleeping time’ is the time from when ‘Stand by Me’ recognizes the start time of sleep to when the "awake" button is pressed on page 6.2. If the calculated ‘sleeping time’ reaches the ‘target sleep time’, a success icon is created on that date, and if not achieved, a failure icon is created. On the other hand, if the ‘Stand by Me’s 'sleep mode' is not used, an icon is not created and appears as a blank. On the right, the percentage of sleep targets achieved over a month is shown as a percentage.






\end{document}
